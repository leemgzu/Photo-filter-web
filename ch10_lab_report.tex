\input opmac
\input ../hisyntax
\tit CS116 Ch10p06-GUI
\sec Lab Description 

In this lab, we need to design a photo effect program that allows users to select a photo and a photo effect. 
After submitting the selections,
the next page will show the corresponding photos.
There are a base class Effect and its three derived classes: Sunset, Grayscale and Invert.
The three derived classes have inherited the function update\_pixel from Effect and overridden the function process.
The lab is meant to practice the polymorphism concepts in chapter 10.
In order to allow the derived class to access the base class data member, 
I use "protected" insead of "private."
To access the objects from different classes in a class hierarchy, 
I set up a vector of pointers to store the address of each object and 
create virtual functions to further access the overriden member function in derived classes.
I reused the code of parsing query string from previous lab and make it a helper function in the photoEffect.cpp file.
I also reimplemented the code from imagemod.cpp in the bigc3code to open and save the bmp files.
In the first version, I did not use the "dirent" structure and directly outputed the image to html.  
The way I did in the previous version is static. 
However, in the real world situations, the dynamically creating html is much more prefered. 
To acheive that, I created another file to generate the html code and modified the code from chapter 11 worked example 1
To enhence the user experience of the program, I also added a back botton.
Last but not the lease, I chose eight photos I took in Yellowstone National Park this summer in te program.
Since the original size is too big to process, I resized them to the 10\% of the origianl.


\filbreak
\sec The Images in Photo Gallery 
\picw=4in \inspic original_photo/bison.jpg
\picw=4in \inspic original_photo/prism.jpg
\picw=4in \inspic original_photo/hikingboots.jpg
\picw=4in \inspic original_photo/hollylake.jpg
\picw=4in \inspic original_photo/elk.jpg
\picw=4in \inspic original_photo/sapphire.jpg
\picw=4in \inspic original_photo/thermogeyser.jpg
\picw=4in \inspic original_photo/mudgeyser.jpg



\sec The Showcase of The Program
The Web Page of the Photo Program
\nl
\picw=5in \inspic webpage1.png
\nl
\filbreak
\picw=5in \inspic webpage2.png
\filbreak

The Next Page After Applying Effect 
\nl
\picw=4in \inspic bisonWithEffect.png
\nl
I also made a video demostrating how the program works. 

\filbreak
\sec Source Code of Effect.h

\hisyntax{C}
\verbinput (-) Effect.h

\filbreak
\sec Source Code of Effect.cpp

\hisyntax{C}
\verbinput (-) Effect.cpp

\filbreak
\sec Source Code of Sunset.h

\hisyntax{C}
\verbinput (-) Sunset.h

\filbreak
\sec Source Code of Sunset.cpp

\hisyntax{C}
\verbinput (-) Sunset.cpp

\filbreak
\sec Source Code of Grayscale.h

\hisyntax{C}
\verbinput (-) Grayscale.h

\filbreak
\sec Source Code of Grayscale.cpp

\hisyntax{C}
\verbinput (-) Grayscale.cpp

\filbreak
\sec Source Code of Invert.h

\hisyntax{C}
\verbinput (-) Invert.h

\filbreak
\sec Source Code of Invert.cpp

\hisyntax{C}
\verbinput (-) Invert.cpp

\filbreak
\sec Source Code of direntry.h

\hisyntax{C}
\verbinput (-) direntry.h

\filbreak
\sec Source Code of htmlGenerator.cpp

\hisyntax{C}
\verbinput (-) htmlGenerator.cpp

\filbreak
\sec Source Code of photoEffect.html

\hisyntax{C}
\verbinput (-) photoEffect.html

\filbreak
\sec Source Code of photoEffect.cpp

\hisyntax{C}
\verbinput (-) photoEffect.cpp



\bye